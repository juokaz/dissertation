\chapter{Conclusions}

Links takes a very different approach to web programming than both PHP and Ruby. Obviously a big part of that is decided by the fact that it's a functional language, thus a lot of paradigms are different and lack of classes and objects makes implementing quite a lot of functionality impossible. Nonetheless, it showed quite a few of interesting features and some of the not so good parts of Links can be improved by extending what a language can do. Trying to improve the language wasn't in the scope of this project, so this was not done, but the recommendations provided could be taken further.

First part of the project was to analyse the language itself and how it does some of the core functions - requests handling, state transfer and forms handling, with obviously language and type system review before. This showed a key difference in that Links web application is defined in terms of functions and overall flow of the application resembles that of a desktop application. Where both PHP and Ruby behaved exactly like a HTTP protocol would, being completely stateless and starting execution from the start every time a new request would come in.

Forms handling was excellent and really made developing applications easier. State transfer mechanism was to include state-describing encoded string as part of the request URL, which worked great for supporting multi-window browsing, although that's not really a big problem for a lot of users and there are ways to avoid this in other languages. Sadly, to support this Links had to make URLs auto-generated and thus rendered bookmarking or search-engine optimisation impossible.

Next and the most disappointing part was about REST. RESTful applications are becoming more and popular as people realize a value in building applications which can scale in terms of both physical and semantic manners. However lack of support for defining custom URLs for functions meant that Links would generate random-looking URLs which were hiding the hierarchy of resources. Secondly, Links didn't provide any access to request information, thus it was impossible to reach to different mime type requests. Which is one of the core concepts behind Hypermedia, which itself is a requirement for RESTful applications. Hence Links cannot be used for RESTful applications.

Database support worked well and the abstraction reminded of using ORMs in Ruby for example, however it didn't really provide anything more than basic access, thus it really limited what could potentially be implemented. Especially the querying part was very simple and using for anything other than simple queries would result in very inefficient SQL queries to be generated (or a lot of SQL queries to be generated). 

While installing Links can be challenging, it's easily deployable once that's done, matching the speed of PHP and Ruby. However web server interactions are still using ancient \textit{cgi-bin} handler, thus applications are limited at how fast they can work. At the same time, implementing a basic layouting/templating system was easy enough by creating a layout function and a few widgets. Even though this wasn't a full fledged templating system, it was more than enough to handle various applications as only so many of them need inheritance and other more complicated features out of templating libraries.

Overall, Links has an interesting take on web programming. While some of it's features make it impossible to use for production-level applications, quite a few concepts of Links are really good and ideally should be also adapter by other languages. And then it comes to cutting-edge web applications - NoSQL and REST are current cutting-edge nodes, both of which are in no way supported by Links, so while Links can do AJAX just fine and thus can be used for some Web 2.0 applications, it has lost it's edge as the industry has advanced.