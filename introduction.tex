\chapter{Introduction}

\section{Cutting-edge Web 2.0 applications}

Let's start with describing what Web 2.0 is. This term was created, by chance, by Tim O'Reilly in a conference session in 2004 ("I first heard the phrase 'Web 2.0' in the name of the Web 2.0 conference in 2004", \citep{web20}). It doesn't mean any technological changes, nor design changes, even though few years later Web 2.0 was stared to be used in describing modern-looking websites, usually identified by a logo having a shadow or a beta/alpha badge, what it means is a change in how web is perceived. Rather than web being just about displaying text and rich media like images, Web 2.0 is all about interactive applications and collaboration. 

Best example of a web application allowing collaboration is forum software. A forum is a certain website, where users can create their profiles and participate in discussions on topics they find interesting. More or less, a forum is a virtual discussion amongst distributed people, with history of their messages being recorded. This was available even before Web 2.0, but the problem was that it was only available in a tiny percentage of websites - dedicated websites. All other websites were static and didn't allow any interaction. This wasn't good any more, with more and more websites being created every day, still users had no way to interact with them, which was something everyone wanted as that would allow to enrich the content of various websites. 

Interactions started rising with first introduction of commenting systems and later with websites allowing to contribute to content. Allowing users to comment on certain parts of ones website, being a news article or a product, allowed users to leave notes, comments or feedback, which then can be read by others, thus creating richer content, extending it further than what original author has planned, and most important - allowing users to contribute content. Later this content contribution started growing new ideas and WikiPedia.org was born\citep{wikipedia}. Currently one of the biggest websites in the whole world, while all content, website itself and platform is being created and maintained by volunteers. 

This was a paradigm shift in semantics of web, this report doesn't focus on that much, however Web 2.0 also "introduced" or in other words started pushing towards rich user interfaces. Rather than a website being a static page, which has links and forms linking to other pages, developers started to look for ways to make web applications behave closer to desktop applications. Which were always more responsive and functional that websites. AJAX was created - Asynchronous JavaScript and XML, a collective group of technologies which utilise JavaScript client-side and XmlHttpRequest (XHR) API to establish asynchronous connections to a server-side application. 

AJAX allows applications to fetch more data whenever a client site JavaScript application decides to, without a user issuing this action. This allows to update certain parts of pages without refreshing the whole application, hence a web mail application can be constantly updating Inbox's list of messages rather than user having to refresh the page to know if he got any more mail. Taken further, applications like Google Maps (http://maps.google.com/) could be built, allowing to interact with a dataset bigger than would normally be consumed a user, by loading only the parts of map which are required to be visible. But since they are loaded asynchronously, with a fast-enough broadband connection, user could zoom in to a map and immediately get a higher detail map.

\section{Existing frameworks and languages}

Links is obviously not the only platform to create web applications, currently PHP, Ruby and Python are all highly popular. Java and .Net solutions are more popular in enterprise segment and they tend to solve problems in a different way, thus they are not going to be included in this report. Out of PHP, Ruby and Python, I chose to focus on two: PHP and Ruby. Python and Ruby are quite similar, both in terms of a language and what they offer in frameworks, thus I chose the more popular, Ruby. PHP is by far the most popular language when it comes to creating websites\citep{popular-php}, thus it couldn't not have been included in this report. 

PHP has a somewhat interesting story about how it became so popular. It is a dynamic language, with a syntax mixed from both Perl and C, however lacking a clear structure in syntax and development practices. It's not uncommon for developers to rate it as one of the worst languages out there\citep{worst-php}, however in 15 years of existence it rose to become the backbone of web. The reason for that is how quickly it is to learn, create a website and have it running deployed locally or on a server. That brought a lot of young developers, which then later moved to other languages or stayed with PHP. 

Some of the world's biggest websites are run powered by PHP (http://www.php.net). In my belief, Links is trying to solve a similar problem which is so great about starting with PHP - starting from zero and having a website running in the shortest amount of time possible. PHP solves this by having a language which is easy to develop in, doesn't take much effort to deploy and see it running, while Links aims for more of an integrated development environment, bringing all three tiers\citep{three-tier} under one language. Roughly 5 years ago, frameworks started rising in PHP, making some of the web-specific functionality even easier, because even though PHP is mainly used for websites it is still a generic language. Thus including quite a few of Links-like behaviours. 

Ruby is also an old language, used heavily for scripting, however it rose as web-ready one, mainly because of Ruby On Rails (http://rubyonrails.org/) framework. After RoR, short for Ruby on Rails, a few other smaller frameworks were created, however the spirit of RoR is still seen in them - creating websites in the less amount of code possible and providing simple solutions to hard problems. Sometimes that created scalability issues\citep{twitter-ror}, however for anything not as big as world scale applications, Ruby solutions worked great. One of the common things in Ruby community is videos like "How to create a blog application in 15 minutes"\citep{ror-blog}, showing off the speed of scaffolding and code generation-powered tools. 

Both Ruby and PHP are really different languages from Links, thus the analysis of them and how that affects development of web applications, is included in one of the sections of this report. 

\section{Report Aims}

This report evaluates Links against two other languages, PHP and Ruby, to asses how usable it is for creating modern web applications. When Links is not capable of doing a certain task, it is evaluated on how difficult and challenging it would be to implement that, assuming that inclusion would be relevant to the goals Links is trying to achieve. With some of the ideas briefly explained above, a list of topics to be analysed in is this:

\begin{description}
  \item[Comparing environments] \hfill \\
  Analysis of how Links, Ruby and PHP handle web applications both in terms of language constructs like type systems and also in terms of web-specific functions like forms
  \item[REST] \hfill \\
  Constrains of REpresentational State Transfer applied to Links applications
  \item[Databases] \hfill \\
  Analysis of how Links, Ruby and PHP handle databases and how they help or don't with certain tasks like database schema management
  \item[Templating] \hfill \\
  Analysis of how Links, Ruby and PHP handle HTML creation and layered templating
  \item[Deployment] \hfill \\
  Deployment process of Links, Ruby and PHP
\end{description}

To come to conclusions and comparisons, a few applications were developed to showcase a certain list of functionality, the key one being a blog application, developed in all three languages. This application allows to compare how different languages approach same or similar problems and evaluate which one is subjectively better. However the goal of this project wasn't to build a full application and then evaluate it by itself, the idea behind this was to compare it based on the language itself based on previous experience, knowledge and research of two other languages.