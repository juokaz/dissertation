%%%%%%%%%%%%%%%%%%%%%%%%
%% Sample use of the infthesis class to prepare a thesis. This can be used as 
%% a template to produce your own thesis.
%%
%% The title, abstract and so on are taken from Martin Reddy's csthesis class
%% documentation.
%%
%% MEF, October 2002
%%%%%%%%%%%%%%%%%%%%%%%%

%%%%
%% Load the class. Put any options that you want here (see the documentation
%% for the list of options). The following are samples for each type of
%% thesis:
%%
%% Note: you can also specify any of the following options:
%%  logo: put a University of Edinburgh logo onto the title page
%%  frontabs: put the abstract onto the title page
%%  deptreport: produce a title page that fits into a Computer Science
%%      departmental cover [not sure if this actually works]
%%  singlespacing, fullspacing, doublespacing: choose line spacing
%%  oneside, twoside: specify a one-sided or two-sided thesis
%%  10pt, 11pt, 12pt: choose a font size
%%  centrechapter, leftchapter, rightchapter: alignment of chapter headings
%%  sansheadings, normalheadings: headings and captions in sans-serif
%%      (default) or in the same font as the rest of the thesis
%%  [no]listsintoc: put list of figures/tables in table of contents (default:
%%      not)
%%  romanprepages, plainprepages: number the preliminary pages with Roman
%%      numerals (default) or consecutively with the rest of the thesis
%%  parskip: don't indent paragraphs, put a blank line between instead
%%  abbrevs: define a list of useful abbreviations (see documentation)
%%  draft: produce a single-spaced, double-sided thesis with narrow margins
%%
%% For a PhD thesis -- you must also specify a research institute:
% \documentclass[phd,ilcc,twoside]{infthesis}

%% For an MPhil thesis -- also needs an institute
% \documentclass[mphil,ianc]{infthesis}

%% MSc by Research, which also needs an institute
% \documentclass[mscres,irr]{infthesis}

%% Taught MSc -- specify a particular degree instead. If none is specified,
%% "MSc in Informatics" is used.
% \documentclass[msc,cogsci]{infthesis}
% \documentclass[msc]{infthesis}  % for the MSc in Informatics

%% Master of Informatics (5 year degree)
% \documentclass[minf]{infthesis}

%% Undergraduate project -- specify the degree course and project type
%% separately
\documentclass[bsc,frontabs]{infthesis}
\course{Computer Science}
\project{Fourth Year Project Report}

%% Put any \usepackage commands you want to use right here; the following is 
%% an example:
\usepackage[numbers]{natbib}

\newenvironment{codelisting}
{\begin{list}{}{\setlength{\leftmargin}{1em}}\item\scriptsize\bfseries}
{\end{list}}

\newenvironment{codetinylisting}
{\begin{list}{}{\setlength{\leftmargin}{1em}}\item\tiny\bfseries}
{\end{list}}

%% Information about the title, etc.
\title{Cutting-Edge Web: Building Web 2.0 Applications In Links}
\author{Juozas Kaziukenas}

%% If the year of submission is not the current year, uncomment this line and 
%% specify it here:
% \submityear{1785}

%% Optionally, specify the graduation month and year:
% \graduationdate{February 1786}

%% Specify the abstract here.
\abstract{%
    This report presents an evaluation of Links, a functional language for web programming combining HTML, JavaScript and SQL handling under one language. A few sample applications were created and language syntax was studied to compare it against two other very popular languages used for web programming - Ruby and PHP. Links behaved very different from the other two, really shining in form handling and ability to implement templating, however it's state handling mechanism while allowing multi-browser browsing more easily, was found to be breaking the overall usability of a language because it would generate unpredictable and unchangeable URLs. Furthermore, database abstraction was easier to use than especially PHP, but lacked features and performance. Most disappointingly, it didn't allow any key RESTful behaviours, thus in the end making it not suitable for developing cutting-edge applications.
}

%% Now we start with the actual document.
\begin{document}

%% First, the preliminary pages
\begin{preliminary}

%% This creates the title page
\maketitle

%% Next we need to have the declaration.
\standarddeclaration

%% Finally, a dedication (this is optional -- uncomment the following line if
%% you want one).
% \dedication{To my mummy.}

%% Create the table of contents
\tableofcontents

%% If you want a list of figures or tables, uncomment the appropriate line(s)
% \listoffigures
% \listoftables

\end{preliminary}

%%%%%%%%
%% Include your chapter files here. See the sample chapter file for the basic
%% format.

\chapter{Introduction}



\section{Cutting-edge Web 2.0 applications}

Let's start with describing what Web 2.0 is. This term was created, by chance, by Dale Dougherty in a conference session in 2005[]. It doesn't mean any technological changes, nor design changes, even though few years later Web 2.0 was stared to be used in describing modern-looking websites, usually identified by a logo having a shadow[] or a beta/alpha badge, what it means is a change in how web is perceived. Rather than web being just about displaying text and rich media like images, Web 2.0 is all about interactive applications and collaboration. 

Best example of a web application allowing collaboration is forum software. A forum is a certain website, where users can create their profiles and participate in discussions on topics they find interesting. More or less, a forum is a virtual discussion amongst distributed people, with history of their messages being recorded. This was available even before Web 2.0, but the problem was that it was only available in a tiny percentage of websites - dedicated websites. All other websites were static and didn't allow any interaction. This wasn't good any more, with more and more websites being created every day, still users had no way to interact with them, which was something everyone wanted as that would allow to enrich the content of various websites. 

Interactions started rising with first introduction of commenting systems and later with websites allowing to contribute to content. Allowing users to comment on certain parts of ones website, being a news article or a product, allowed users to leave notes, comments or feedback, which then can be read by others, thus creating richer content, extending it further than what original author has planned, and most important - allowing users to contribute content. Later this content contribution started growing new ideas and WikiPedia was born[]. Currently one of the biggest websites in the whole world, while all content, website itself and platform is being created and maintained by volunteers. 

This was a paradigm shift in semantics of web, this report doesn't focus on that much, however Web 2.0 also "introduced" or in other words started pushing towards rich user interfaces. Rather than a website being a static page, which has links and forms linking to other pages, developers started to look for ways to make web applications behave closer to desktop applications. Which were always more responsive and functional that websites. AJAX was created - Asynchronous Javascript and XML, a collective group of technologies which utilise Javascript client-side and XmlHttpRequest (XHR) API to establish asynchronous connections to a server-side application. 

AJAX allows applications to fetch more data whenever a client site JavaScript application decides to, without a user issuing this action. This allows to update certain parts of pages without refreshing the whole application, hence a web mail application can be constantly updating Inbox's list of messages rather than user having to refresh the page to know if he got any more mail. Taken further, applications like Google Maps[] could be built, allowing to interact with a dataset bigger than would normally be consumed a user, by loading only the parts of map which are required to be visible. But since they are loaded asynchronously, with a fast-enough broadband connection, user could zoom in to a map and immediately get a higher detail map.

\section{Existing frameworks and languages}

Links is obviously not the only platform to create web applications, currently PHP, Ruby and Python are all highly popular. Java and .Net solutions are more popular in enterprise segment and they tend to solve problems in a different way, thus they are not going to be included in this report. Out of PHP, Ruby and Python, I chose to focus on two: PHP and Ruby. Python and Ruby are quite similar, both in terms of a language and what they offer in frameworks, thus I chose the more popular, Ruby. PHP is by far the most popular language when it comes to creating websites[], thus it couldn't not have been included in this report. 

PHP has a somewhat interesting story about how it became so popular. It is a dynamic language, with a syntax mixed from both Perl and C, however lacking a clear structure in syntax and development practices. It's not uncommon for developers to rate it as one of the worst languages out there[], however in 15 years of existence it rose to become the backbone of web. The reason for that is how quickly it is to learn, create a website and have it running deployed locally or on a server. That brought a lot of young developers, which then later moved to other languages or stayed with PHP. 

Some of the world's biggest websites are run powered by PHP[]. In my belief, Links is trying to solve a similar problem which is so great about starting with PHP - starting from zero and having a website running in the shortest amount of time possible. PHP solves this by having a language which is easy to develop in, doesn't take much effort to deploy and see it running, while Links aims for more of an integrated development environment, bringing all three tiers[] under one language. Roughly 5 years ago, frameworks started rising in PHP, making some of the web-specific functionality even easier, because even though PHP is mainly used for websites it is still a generic language. Thus including quite a few of Links-like behaviours. 

Ruby is also an old language, used heavily for scripting, however it rose as web-ready one, mainly because of Ruby On Rails[] framework. After RoR, short for Ruby on Rails, a few other smaller frameworks were created, however the spirit of RoR is still seen in them - creating websites in the less amount of code possible and providing simple solutions to hard problems. Sometimes that created scalability issues[], however for anything not as big as world scale applications, Ruby solutions worked great. One of the common things in Ruby community is videos like "How to create a blog application in 15 minutes"[], showing off the speed of scaffolding and code generation-powered tools. 

Both Ruby and PHP are really different languages from Links, thus the analysis of them and how that affects development of web applications, is included in one of the sections of this report. 

\section{Web and REST}

Big part of modern applications' requirements is how RESTful they are. REST[], REpresentational State Transfer, was defined in 2000 by Roy Fielding, as a style of application architecture bound by these constraints:

\begin{description}
  \item[Client–server] A uniform interface separates clients from servers, clients are not concerned with data storage and servers are not concerned with the user interface or user state, so that servers can be simpler and more scalable
  \item[Stateless] No client context is being stored on the server between requests. Each request from any client contains all of the information necessary to service the request, and any session state is held in the client
  \item[Cacheable] Clients can cache responses. Responses must therefore, implicitly or explicitly, define themselves as cacheable, or not, to prevent clients reusing stale or inappropriate data in response to further requests
  \item[Layered system] A client cannot ordinarily tell whether it is connected directly to the end server, or to an intermediary along the way
  \item[Code on demand (optional)] Servers are able temporarily to extend or customize the functionality of a client by the transfer of executable code
  \item[Uniform interface] The uniform interface between clients and servers simplifies and decouples the architecture, which enables each part to evolve independently
\end{description}

When a web application is called RESTful, it has a certain amount of resources, all identified by a global identifier, which can be manipulated over a standardized interface, and exchange representations of them in a format accepted by a client, assuming there exists one. Even though REST style doesn't define this as being HTTP-only, for our purposes, let's assume that all definitions below, are related to HTTP, the protocol web applications talk in. Web application resources would be identified by their URIs, the interface allowing to manipulate them is obviously HTTP and representations would come as different text types.

It thus expected that an interface would behave like this:

\begin{codelisting}
\begin{verbatim}
$ curl -H "Accept: application/json" -X POST -d "name=iphone&title=Iphone5" http://localhost/products
{'name':'Iphone5', 'location': 'http://localhost/products/iphone'}

$ curl -H "Accept:application/html" http://localhost/products/iphone
<html><body>Iphone5</body></html>

$ curl -H "Accept:application/xml" http://localhost/products/iphone
<product><name>Iphone5</name></product>

$ curl -H "Accept:application/json" http://localhost/products/iphone
{'name':'Iphone5'}

$ curl -H "Accept:text/plain"localhost/products/iphone
Iphone5
\end{verbatim}
\end{codelisting}

Even though a usual web application in most cases is just a website, thus it has only representation - HTML - I would expect a modern web application framework or a language to support RESTfullness, as it solves quite a lot of problems in current web state. First of all, it allows any website to become machine readable easily, just by exchanging a different representation, say XML, than the one returned to browsers. Second, the way it defines constrains for cacheability, statelessness and layered systems transparency, it makes web applications scale better, both in terms of development speed and ease and also performance. 

Most importantly, I think Hypermedia[] is crucial part of web as a whole, and for it to evolve for years to come, all web endpoints, being simple websites or machine consumable APIs, have to support it. It is not a big requirement for a certain web framework or a language, there are only two things it needs to do well - allow to expose resources identifiable by unique IDs, which should be predictable and reliable, and handle different response types, not just HTML. 

\section{Report Aims}

%% Sample chapter file, for use in a thesis.
%% Don't forget to put the \chapter{...} header onto each file.

\chapter{Comparing environments}

\section{Type systems and implication}

Links is a strongly typed language, thus even if a variable is defined as \textit{var i = 1}, at compile time this would be interpreted as \textit{i} variable having a type of \textit{int} and a value of \textit{5}. Furthermore, functions can have specifications of the types they accept. Both PHP and Ruby on the other hand are greatly different, both using weak-typing. Variable types are not checked in compile type and not necessary even at execution, which might result in unexpected behaviours when two different types are used in for example math operations. Both languages would try to convert to a type would allow that operation, but execution of statements like \textit{"one" + 1} is always unpredictable for a developer.

One of the key ideas behind Links is solving impedance mismatch problem[], which can be briefly described as an issue for a developer when he needs to map different types to and from different languages. Most web applications consist of at least a server-side language, JavaScript and SQL, which all have different type systems. For example all database engines have different date and time types, thus there should be some level of logic which knows how to parse them to be represented in a language-level way. And once a user enters a date in a form, some code should convert that to language-level type which again would have to be converted to a database specific type.

Database types are further discussed in the Database section, but the implication of Links type system and the fact that it itself generates both JavaScript and SQL code, means that a developer only needs to worry about the types he knows, which are language types. Where in PHP one would have to convert a datetime string to a \textit{DateTime} object and then handle that for display purposes, Links would handle that for him. This improves development time vastly and also because it's a compile time check, it means that it's impossible to make a mistake such as storing a \textit{String} type value in \textit{int} column.

However strongly typed languages are not necessary ideal for the web. The problem with the web, is that a developer has no control of what user is doing. When one is presented with a form, he might choose to enter his birthday as a UNIX timestamp, like \textit{1333449313}, and thus type matching would implicitly fail as this is not the datetime type an application was expecting. Obviously this is solvable by a validation stage, but this is much easier in PHP/Ruby as those would gladly accept any value which a developer can check by himself, where in Links the application would just fail to accept it as it doesn't match the definition of the type.

\section{Language architecture}

To start with, Links works differently than both Ruby and PHP in the most fundamental way. While these two are executed to result in some output or instantiate a process, Links works also as a code generator, generating JavaScript and SQL code from Links source-code. This is a feature, which PHP/Ruby have no comparable solution and thus is a major difference. Some of the database integration can be replicated by ORMs, as discussed in Databases section, but converting Links execution flows to JavaScript functions in no way cannot be achieved. This is a compiled level behaviour and even extensions of PHP/Ruby would not be able to do this as it would require a complete interception of the core. 

PHP and Ruby are both imperative object-oriented languages, with especially Ruby using objects for everything and even primitives like integers being objects, while Links is a functional language. Thus coding style is hugely different, which then leads to different approach to most of the web programming problems. Because of the functional nature, as in analysed below in more detail, web application is constructed as a graph of function calls, removing the boundaries of HTTP, where other languages assume each request must start and end somewhere same (dispatcher).

Because Links is a functional language, it doesn't have any global state, as part of the design, however sometimes a lack of it is extremely noticeable when migrating Ruby or PHP applications to Links. In PHP for example, one can always access state information by accessing \textit{\$\_SESSION} global array, same for request parameters which are accessible through \textit{\$\_GET} and \textit{\$\_POST}. While not necessary required to build an application, passing around various properties in a bigger application might become cumbersome and increase complexity. 

\section{Requests handling}

Links handles HTTP requests very differently from say Ruby or PHP - it makes a developer think as if HTTP was a stateful protocol. So rather than a link point to another script or handler, in Links a link would \textit{point} to a function call, which is parsed by Links parser and an actual URL is generated to support this functionality. The approach of links or forms being pointed to functions rather than external scripts resembles a desktop application, where one process handles state transfers. Here are 3 examples of anchor links in different languages:

\begin{codelisting}
\begin{verbatim}
// Ruby (Sinatra)
<a href="/posts/#{p.id}" class="entry-title" rel="bookmark">#{p.title}</a>
// PHP
<a href="/posts.php?id=<?php echo $p->id ?>" class="entry-title" rel="bookmark"><?php echo $p->title ?></a>
// Links
<a l:href="{showPost(id)}" class="entry-title" rel="bookmark">{stringToXml(title)}</a>
\end{verbatim}
\end{codelisting}

In PHP and Ruby case, the URLs it points at will have to handle a new request coming in without knowing anything from a previous request, thus they are stateless, as HTTP was designed to be. Since Links' link points to a function, it can only execute a new request if it knows the function to be executed and it's arguments, which requires Links to generate an encoded URL in a form of \textit{'blog.links?\_k=BAH7BgH6BgQB4wsB 4gsAAAFz4wv7Bg=='}, where \textit{\_k} query parameter contains that information. Of course assuming that \textit{showPost()} is to be executed server side, as Links also allows functions to be executed client side just by annotating them with a \textit{client} keyword, which makes it generate JavaScript code for that function.

To support these URLs, different framework handle matching differently. MVC architecture, described below, would have a routing component which maps a list of routes like above to specific controllers' actions. Actions themselves would handle different HTTP verbs or this would be part of route definition. Sintara framework, similarly to a lot of small web frameworks, solves this problem by directly matching a route and verb combo to a certain handler, which is called assuming that combo is satisfied. Example below shows a Sintara application blueprint which internally just modifies Sinatra's routing list to add new string routes to match against:

\begin{codelisting}
\begin{verbatim}
get '/products/:id*' do
   "Iphone"
end

post '/products/:id*' do
   id = params[:id]
   # Handle update
end

delete '/products/:id*' do
   id = params[:id]
   # Delete delete by id
end
\end{verbatim}
\end{codelisting}

This allows creating new resources really quickly and quite clearly indicates what requests it accepts. PHP as a language doesn't help with this in any way, because of the nature of it being just a script, where Ruby applications are small web-servers by themselves. In case of PHP, a web-server of choice must redirect all requests matching non-files (so images, CSS and JavaScripts can still be accessed) to one handler script which parses the passed route to decide a handler to execute. Most of PHP frameworks do this out of the box, usually by employing a routing mechanism described above and redirection of requests to a handling script is achieved by tools like \textit{mod\_rewrite}[] in Apache.

In the end, differently from other languages, Links doesn't have a \textit{request --\textgreater  response} based model, which is especially visible in PHP, where a request would create a new instance of PHP process and whole script will be executed once for that request, thus completing the cycle. In Links, users move between states by clicking on links or submitting forms which in turn modify a state of application using the encoded information in query string. This makes it really easy to start creating applications as it doesn't require to understand that HTTP is stateless and execution flow is different from say Java desktop applications.

\subsection{MVC}

Most of modern web applications are written using MVC[], Model-View-Controller, architecture based frameworks. Even though they also come from desktop applications and don't really apply to web applications directly, developers managed to figure out a way to make them work. In most cases the architecture is like this:

\begin{description}
  \item[Model] \hfill \\
  Part of application responsible for storing and retrieving data \\
  \textbf{Calls}: nothing \\
  \textbf{Called by}: Controller
  \item[View] \hfill \\
  Part of application responsible for generating representation of data in for example HTML \\
  \textbf{Calls}: Model \\
  \textbf{Called by}: Controller
  \item[Controller] \hfill \\
  Part of application responsible for handling request data, retrieving data from a model(-s), passing that a view(-s) and returning a response \\
  \textbf{Calls}: Model, View \\
  \textbf{Called by}: nothing
\end{description}

This is not possible to implement in Links as it massively differs from it's way of defining application logic and also because MVC is in most cases implemented using object-oriented languages. However this is worth mentioning as a point of reference, since this is the direction the industry is moving towards, although this would require introduction of objects into Links language which would be impractical and likely too difficult.

In MVC architecture, each Controller has a list of \textit{actions} which define handlers in a similar fashion as the ones above in Sintatra example. Because of this layered architecture (\textit{N} controllers have \textit{M} actions), it is possible to access certain parts of functionality without an explicitly defined routing configuration, by just relying on a convention that \textit{/books/view} URL would execute the \textit{view} action in \textit{books} controller. Because of this, MVC architecture allows new functionality to be added quickly, without having to make any other modifications other than adding new actions. Which is also true for Links, as it doesn't require anything else than to create a function.

\section{State transfer}

Web applications do not have states per-say, not usually, but rather need to remember various information like products placed in a cart. Most of server-side programming languages or frameworks employ a notion of sessions, a collection of information stored server-side, which is then accessible to application by matching session ID stored in a user's request to the storage. This session ID is passed between requests either as query string parameter or as a cookie. Cookies are used because they are domain-locked, thus only if a user is visiting the same domain the browser would pass that cookie, this secures against cookies leaking to malicious websites. 

However this model, even though it's the most popular, has problems. First of all, opening multiple browser windows would still use the same session, which might cause unexplained effects if different windows are being tried to be used as different instances of an application. Now if this is a desirable behaviour depends on an application, but even using back/forward buttons might result in an error, as clicking back doesn't revert the state server-side. Secondly, user has no way, unless the website provides this functionality externally as a login page for example, to transmit his state from one machine to another.

Links embeds state information within a page itself as an encoded serialized state of the program transferred between different pages as a query parameter \textit{\_k}. By doing this Links applications are maintaining state on per-windows basis rather than per-user basis as compared to say PHP. Thus the user can open as many browser windows as he requires and navigate between them freely without modifying the state of any other window. Furthermore, clicking back/forward buttons would not cause unexpected behaviour as it would in some PHP or Ruby applications and any single URL always points to a specific state thus it can be always used to go back to it, even from a different machine. This, again, makes Links look and behave very much like a desktop application, removing focus on web applications as having separate pages.

State information is only encoded using \textit{Base64} algorithm, which can be used to decode the string, modify the state information and after encoding it again use it to move to a created state. This is a security risk, as an attacker can reach any state of the application and potentially gain access to information which should not be accessible publicly. PHP and Ruby are not that hard to trick, because only session ID is stored in a client request and modifying that, if it's not part of a URL (which is a recommendation), is only possible with a middle-man attack. Encrypting the state information with an application hash, configured explicitly or generated automatically, would be more suitable.

\begin{description}
  \item[wine.links?\_cont=AtEUA7gOAAADATK5DgAAAwM4NDi6DgAAAwItMfoUAA==] \hfill \\
  2 - wine ID \\
  878 - cart ID \\
  -1 - order ID (order doesn't exist yet)
  \item[wine.links?\_cont=AtEUA7gOAAADAjEwuQ4AAAMDODQ4ug4AAAMCLTH6FAA=] \hfill \\
  10 - wine ID \\
  878 - cart ID \\
  -1 order ID (order doesn't exist yet)
  \item[wine.links?\_k=BAH7BgH6BgQBxRECrw4AAAMDODQ4sA4AAAMCLTHFEfsG] \hfill \\
  878 - cart ID \\
  -1 - order ID (order doesn't exist yet)
\end{description}

After a simple analysis using a Hex editor and \textit{Base64} decrypting, I was able to reverse engineer logic behind the wine store application provided as part of Links examples. Enough time to figure out how a function is chosen for a specific URL wasn't invested, although these hashes contain more information than extracted above, thus that has to be the way. Hex editor clearly showed the arguments passed to each function and quite easily I was able to modify a wine ID to produce a new URL which would allow me to add a new wine to my shopping cart. However changing cart ID was possible thus allowing to read/modify other people's carts.

Even though there are problems with Links approach, it scales better for distributed systems. Because user's state is not stored server-side, but rather as part of a request, multiple instances of application in different physical machines should always be able to handle that request. Most PHP applications would start with sessions stored locally on a disk, but as soon as they need to handle more load thus get additional servers, they have to modify the application to store sessions in a shared instance, for example MemCache[]. Thus Links obeys share-nothing[] architecture from the start, while other languages in this report would need additional work to make them behave in a scalable way.

\section{Forms}

One of the most interesting features of Links is form handling. Some of the behaviours of Links are also available in Ruby and PHP frameworks, but they work slightly differently, especially since they do not have strongly typed variables, where in Links a formlet would yield an exact type it was asked to. For example by defining a form like this, it would finally return an \textit{Itinerary}-tagged value with two dates inside it:

\begin{codelisting}
\begin{verbatim}
fun dateFormlet(msg) {
  formlet <#>
    {stringToXml(msg)} <br />
    Day:   {inputInt -> day}
    Month: {inputInt -> month}
    Year:  {inputInt -> year}
  </#>
  yields {
    Date(day, month, year)
  }
}

formlet
  <table>
  <tr><td>
    {dateFormlet("Arrival") -> arrival}
  </td><td>
    {dateFormlet("Departure") -> departure}
  </td></tr>
  </table>
yields
  Itinerary(arrival, departure)
\end{verbatim}
\end{codelisting}

When forms need to be handled, their values are automatically passed to a function as arguments. This step alone in say PHP would take building a separate page handling the form and then redirect back to a success page. However with Links it's expressed simply like this:

\begin{codelisting}
\begin{verbatim}
<form l:action="{search(query)}">
  <input type="text" l:name="query" />
  <input type="submit" value="Search" />
</form>
\end{verbatim}
\end{codelisting}

In this case a formlet is not used for a sake of simplicity, however it could be generating the same form layout too. Combined with \textit{sendSuspend} which allows to branch out from a flow to serve a page required to be completed, for example a login page before an order is displayed, forms become a natural part of an application, rather than like in other languages just a piece of HTML sending the request to another page to handle it.

Because Links functions accept to be configured to work either server side or client side, building forms in Links is far better than in any other framework or language as it allows to use the same code to build a form and then switch it to be working using AJAX, without having to recode the HTML markup, JavaScript or any other part of the application. This makes going from basic behaviours to rich user interfaces, as part of Web 2.0, really easy. 
%% Sample chapter file, for use in a thesis.
%% Don't forget to put the \chapter{...} header onto each file.

\chapter{REST}

First you have to get some dead tissue (the knack here is in also managing to
avoid a short custodial sentence for grave-robbing). Then you need one of
those crackling things which makes lots of sparks. The work must also be
conducted in a suitable dark and ancient castle, in close proximity to a town
of highly suspicious and pitchfork-wielding peasants.

\section{What is REST}

\section{URL structures}

\section{State-fullness of Links}

\subsection{Continuous requests (global state)}

\section{Improving Links}

\subsection{URL generation}

\subsection{State control}

%% Sample chapter file, for use in a thesis.
%% Don't forget to put the \chapter{...} header onto each file.

\chapter{Databases}

It's very unusual for a modern web application to be running without a database – unless they work as computation APIs, with which you can query for a specific result which is then computed on-a-fly, web applications read data from a certain data-store and represent that in a human, or machine, readable way.

\section{Databases in Links}

\section{Handling databases in modern frameworks}

\subsection{Ruby}

\subsection{PHP}

\section{Shortcomings of ORMs}

\section{Shortcomings of Links}

Even though databases and tables are defined in an abstract way, which looks really similar to how ORMs would define them, however it's only a one-way definition. Thus meaning that data can be read from a database and mapped to language types, but those definitions cannot be used to manage the database structure itself. This is something which makes it more likely to be prone to errors when a certain database is out of sync with the application definitions for it. Developer is hence required to make sure that every time he changes the database or application, he updates the other one. Theoretically this is not a big problem, however practically, with development team bigger than one developer and with multiple servers being used for development and production, it can become tricky to manage this. 

\section{Slow-SQL generation}

\section{No-SQL databases}

\section{Links improvements}

%% Sample chapter file, for use in a thesis.
%% Don't forget to put the \chapter{...} header onto each file.

\chapter{Templating}

First you have to get some dead tissue (the knack here is in also managing to
avoid a short custodial sentence for grave-robbing). Then you need one of
those crackling things which makes lots of sparks. The work must also be
conducted in a suitable dark and ancient castle, in close proximity to a town
of highly suspicious and pitchfork-wielding peasants.

\section{How templating is used for web applications}

\section{Basic layouts}

\section{Widgets}

\section{Implementing in Links}

\section{Templating frameworks in modern frameworks}

\subsection{Inheritance}

\subsection{Helpers}

\subsection{Implementation in Links}

%% Sample chapter file, for use in a thesis.
%% Don't forget to put the \chapter{...} header onto each file.

\chapter{Deployment}

First you have to get some dead tissue (the knack here is in also managing to
avoid a short custodial sentence for grave-robbing). Then you need one of
those crackling things which makes lots of sparks. The work must also be
conducted in a suitable dark and ancient castle, in close proximity to a town
of highly suspicious and pitchfork-wielding peasants.

\section{Local env setup}

\section{How applications are deployed}

\subsection{PHP}

\subsection{Ruby}

\section{Deploying Links applications}

\section{Improving deployment}

\chapter{Conclusions}

csacsaacs
% \include{chap2}
%% ... etc ...

%%%%%%%%
%% Any appendices should go here. The appendix files should look just like the
%% chapter files.
\appendix
%% ... etc...

%% Choose your favourite bibliography style here.
\bibliographystyle{apalike}

%% If you want the bibliography single-spaced (which is allowed), uncomment
%% the next line.
% \singlespace

%% Specify the bibliography file. Default is thesis.bib.
\bibliography{references}

%% ... that's all, folks!
\end{document}
