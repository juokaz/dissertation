%% Sample chapter file, for use in a thesis.
%% Don't forget to put the \chapter{...} header onto each file.

\chapter{Comparing environments}

\section{Type systems and implication}

Links is a strongly typed language, thus even if a variable is defined as \textit{var i = 1}, at compile time this would be interpreted as \textit{i} variable having a type of \textit{int} and a value of \textit{5}. Furthermore, functions can have specifications of the types they accept. Both PHP and Ruby on the other hand are greatly different, both using weak-typing. Variable types are not checked in compile type and not necessary even at execution, which might result in unexpected behaviours when two different types are used in for example math operations. Both languages would try to convert to a type would allow that operation, but execution of statements like \textit{"one" + 1} is always unpredictable for a developer.

One of the key ideas behind Links is solving impedance mismatch problem[], which can be briefly described as an issue for a developer when he needs to map different types to and from different languages. Most web applications consist of at least a server-side language, JavaScript and SQL, which all have different type systems. For example all database engines have different date and time types, thus there should be some level of logic which knows how to parse them to be represented in a language-level way. And once a user enters a date in a form, some code should convert that to language-level type which again would have to be converted to a database specific type.

Database types are further discussed in the Database section, but the implication of Links type system and the fact that it itself generates both JavaScript and SQL code, means that a developer only needs to worry about the types he knows, which are language types. Where in PHP one would have to convert a datetime string to a \textit{DateTime} object and then handle that for display purposes, Links would handle that for him. This improves development time vastly and also because it's a compile time check, it means that it's impossible to make a mistake such as storing a \textit{String} type value in \textit{int} column.

However strongly typed languages are not necessary ideal for the web. The problem with the web, is that a developer has no control of what user is doing. When one is presented with a form, he might choose to enter his birthday as a UNIX timestamp, like \textit{1333449313}, and thus type matching would implicitly fail as this is not the datetime type an application was expecting. Obviously this is solvable by a validation stage, but this is much easier in PHP/Ruby as those would gladly accept any value which a developer can check by himself, where in Links the application would just fail to accept it as it doesn't match the definition of the type.

\section{Language architecture}

PHP and Ruby are both imperative object-oriented languages, with especially Ruby using objects for everything and even primitives like integers being objects, while Links is a functional language. 

\section{Requests handling}

Links handles HTTP requests very differently from say Ruby or PHP - it makes a developer think as if HTTP was a stateful protocol. So rather than a link point to another script or handler, in Links a link would \textit{point} to a function call, which is parsed by Links parser and an actual URL is generated to support this functionality. The approach of links or forms being pointed to functions rather than external scripts resembles a desktop application, where one process handles state transfers. Here are 3 examples of anchor links in different languages:

\begin{codelisting}
\begin{verbatim}
// Ruby (Sinatra)
<a href="/posts/#{p.id}" class="entry-title" rel="bookmark">#{p.title}</a>
// PHP
<a href="/posts.php?id=<?php echo $p->id ?>" class="entry-title" rel="bookmark"><?php echo $p->title ?></a>
// Links
<a l:href="{showPost(id)}" class="entry-title" rel="bookmark">{stringToXml(title)}</a>
\end{verbatim}
\end{codelisting}

In PHP and Ruby case, the URLs it points at will have to handle a new request coming in without knowing anything from a previous request, thus they are stateless, as HTTP was designed to be. Since Links' link points to a function, it can only execute a new request if it knows the function to be executed and it's arguments, which requires Links to generate an encoded URL in a form of \textit{'blog.links?\_k=BAH7BgH6BgQB4wsB 4gsAAAFz4wv7Bg=='}, where \textit{\_k} query parameter contains that information. Of course assuming that \textit{showPost()} is to be executed server side, as Links also allows functions to be executed client side just by annotating them with a \textit{client} keyword, which makes it generate JavaScript code for that function.

To support these URLs, different framework handle matching differently. MVC architecture, described below, would have a routing component which maps a list of routes like above to specific controllers' actions. Actions themselves would handle different HTTP verbs or this would be part of route definition. Sintara framework, similarly to a lot of small web frameworks, solves this problem by directly matching a route and verb combo to a certain handler, which is called assuming that combo is satisfied. Example below shows a Sintara application blueprint which internally just modifies Sinatra's routing list to add new string routes to match against:

\begin{codelisting}
\begin{verbatim}
get '/products/:id*' do
   "Iphone"
end

post '/products/:id*' do
   id = params[:id]
   # Handle update
end

delete '/products/:id*' do
   id = params[:id]
   # Delete delete by id
end
\end{verbatim}
\end{codelisting}

This allows creating new resources really quickly and quite clearly indicates what requests it accepts. PHP as a language doesn't help with this in any way, because of the nature of it being just a script, where Ruby applications are small web-servers by themselves. In case of PHP, a web-server of choice must redirect all requests matching non-files (so images, CSS and JavaScripts can still be accessed) to one handler script which parses the passed route to decide a handler to execute. Most of PHP frameworks do this out of the box, usually by employing a routing mechanism described above and redirection of requests to a handling script is achieved by tools like \textit{mod\_rewrite}[] in Apache.

In the end, differently from other languages, Links doesn't have a \textit{request --\textgreater  response} based model, which is especially visible in PHP, where a request would create a new instance of PHP process and whole script will be executed once for that request, thus completing the cycle. In Links, users move between states by clicking on links or submitting forms which in turn modify a state of application using the encoded information in query string. This makes it really easy to start creating applications as it doesn't require to understand that HTTP is stateless and execution flow is different from say Java desktop applications.

\subsection{MVC}

Most of modern web applications are written using MVC[], Model-View-Controller, architecture based frameworks. Even though they also come from desktop applications and don't really apply to web applications directly, developers managed to figure out a way to make them work. In most cases the architecture is like this:

\begin{description}
  \item[Model] \hfill \\
  Part of application responsible for storing and retrieving data \\
  \textbf{Calls}: nothing \\
  \textbf{Called by}: Controller
  \item[View] \hfill \\
  Part of application responsible for generating representation of data in for example HTML \\
  \textbf{Calls}: Model \\
  \textbf{Called by}: Controller
  \item[Controller] \hfill \\
  Part of application responsible for handling request data, retrieving data from a model(-s), passing that a view(-s) and returning a response \\
  \textbf{Calls}: Model, View \\
  \textbf{Called by}: nothing
\end{description}

This is not possible to implement in Links as it massively differs from it's way of defining application logic and also because MVC is in most cases implemented using object-oriented languages. However this is worth mentioning as a point of reference, since this is the direction the industry is moving towards, although this would require introduction of objects into Links language which would be impractical and likely too difficult.

In MVC architecture, each Controller has a list of \textit{actions} which define handlers in a similar fashion as the ones above in Sintatra example. Because of this layered architecture (\textit{N} controllers have \textit{M} actions), it is possible to access certain parts of functionality without an explicitly defined routing configuration, by just relying on a convention that \textit{/books/view} URL would execute the \textit{view} action in \textit{books} controller. Because of this, MVC architecture allows new functionality to be added quickly, without having to make any other modifications other than adding new actions. Which is also true for Links, as it doesn't require anything else than to create a function.

\section{State transfer}

Web applications do not have states per-say, not usually, but rather need to remember various information like products placed in a cart. Most of server-side programming languages or frameworks employ a notion of sessions, a collection of information stored server-side, which is then accessible to application by matching session ID stored in a user's request to the storage. This session ID is passed between requests either as query string parameter or as a cookie. Cookies are used because they are domain-locked, thus only if a user is visiting the same domain the browser would pass that cookie, this secures against cookies leaking to malicious websites. 

However this model, even though it's the most popular, has problems. First of all, opening multiple browser windows would still use the same session, which might cause unexplained effects if different windows are being tried to be used as different instances of an application. Now if this is a desirable behaviour depends on an application, but even using back/forward buttons might result in an error, as clicking back doesn't revert the state server-side. Secondly, user has no way, unless the website provides this functionality externally as a login page for example, to transmit his state from one machine to another.

Links embeds state information within a page itself as an encoded serialized state of the program transferred between different pages as a query parameter \textit{\_k}. By doing this Links applications are maintaining state on per-windows basis rather than per-user basis as compared to say PHP. Thus the user can open as many browser windows as he requires and navigate between them freely without modifying the state of any other window. Furthermore, clicking back/forward buttons would not cause unexpected behaviour as it would in some PHP or Ruby applications and any single URL always points to a specific state thus it can be always used to go back to it, even from a different machine. This, again, makes Links look and behave very much like a desktop application, removing focus on web applications as having separate pages.

State information is only encoded using \textit{Base64} algorithm, which can be used to decode the string, modify the state information and after encoding it again use it to move to a created state. This is a security risk, as an attacker can reach any state of the application and potentially gain access to information which should not be accessible publicly. PHP and Ruby are not that hard to trick, because only session ID is stored in a client request and modifying that, if it's not part of a URL (which is a recommendation), is only possible with a middle-man attack. Encrypting the state information with an application hash, configured explicitly or generated automatically, would be more suitable.

Even though there are problems with Links approach, it scales better for distributed systems. Because user's state is not stored server-side, but rather as part of a request, multiple instances of application in different physical machines should always be able to handle that request. Most PHP applications would start with sessions stored locally on a disk, but as soon as they need to handle more load thus get additional servers, they have to modify the application to store sessions in a shared instance, for example MemCache[]. Thus Links obeys share-nothing[] architecture from the start, while other languages in this report would need additional work to make them behave in a scalable way.

\section{Forms}

\section{Missing features of Links}
